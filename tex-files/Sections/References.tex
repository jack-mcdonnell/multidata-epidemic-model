
% References

\begin{thebibliography}{99}
	
%	\bibitem{Abbey 1952} Abbey, H. (1952). An examination of the Reed-Frost theory of epidemics. Human Biology, 24(3), 201–233.
	
%	\bibitem{Bacaer 2011} Bacaër N. (2011) Halley’s life table (1693). In: A Short History of Mathematical Population Dynamics. Springer, London
	
	\bibitem{Ball 1997} Ball F., Mollison D., Scalia-Tomba G. (1997). Epidemics with Two Levels of Mixing. Ann App Prob; 7(1): 46-89. https://doi.org/10.1214/aoap/1034625252.
	
	\bibitem{Bartlett 1956} Bartlett, M.S. (1956) Deterministic and Stochastic Models for recurrent epidemics. Proc. of the Third Berkley Symp. on Math. Stats. and Prob.4 81-108.
	
%	\bibitem{Becker 1981} Becker, N. (1981). A General Chain Binomial Model for Infectious Diseases. Biometrics, 37(2), 251–258. https://doi.org/10.2307/2530415.
	
%	\bibitem{Becker Dietz 1995} Becker, N. G., \& Dietz, K. (1995). The effect of household distribution on transmission and control of highly infectious diseases. Mathematical Biosciences, 127(2), 207–219. https://doi.org/10.1016/0025-5564(94)00055-5.
	
	\bibitem{Begon et al 2002} 
	Begon, Bennett, Bowers, French, Hazel, \& Turner. (2002). A clarification of transmission terms in host-microparasite models: numbers, densities and areas. Epidemiology and Infection, 129(1), 147–153. https://doi.org/10.1017/S0950268802007148.
	
	\bibitem{Bernoulli 1766} Bernoulli, D. (1766) Essai d'une nouvelle analyse de la mortalite causee par la petite verole et des avantages de l'inoculation pour la prevenir. Mem. Math. Phys., Academie Royale des Sciences, Paris, 1760, pp. 1-45.
	
	\bibitem{Bernoulli Blower 2004} Bernoulli, D., \& Blower, S. (2004). An attempt at a new analysis of the mortality caused by smallpox and of the advantages of inoculation to prevent it. Reviews in Medical Virology, 14(5), 275–288. https://doi.org/10.1002/rmv.443.
	
	\bibitem{Black et al 2013} Black AJ, House T, Keeling MJ, Ross JV. (2013) Epidemiological consequences of household-based antiviral prophylaxis for pandemic influenza. J R Soc Interface 10: 20121019. http://dx.doi.org/10.1098/rsif.2012.1019.
	
	\bibitem{Black Ross 2013} Black A.J., Ross J.V. (2013) Estimating a Markovian Epidemic Model Using Household Serial Interval Data from the Early Phase of an Epidemic. PLoS ONE 8(8): e73420. doi:10.1371/journal.pone.0073420.
	
%	\bibitem{Black Ross 2015} Black, A.J., Ross J.V. (2015) Computation of epidemic final size distributions. J. Theor. Biol. 367, 159–165.
	
	\bibitem{Black et al 2017} Black, A.J., Geard, N., McCaw, J.M.,McVernon, J., Ross, J.V. (2017). Characterising pandemic severity and transmissibility from data collected during first few hundred studies. Epidemics 19, 61–73.
	
%	\bibitem{Black 2019} Black, A. (2019). Importance sampling for partially observed temporal epidemic models. Statistics and Computing, 29(4), 617–630. https://doi.org/10.1007/s11222-018-9827-1.
	
%	\bibitem{Brachman 2003} Brachman, P. (2003). Infectious diseases—past, present, and future, International Journal of Epidemiology, Volume 32, Issue 5, Pages 684–686, https://doi.org/10.1093/ije/dyg282.
	
%	\bibitem{Brauer Castillo-Chavez 2012} Brauer, F., \& Castillo-Chavez, Carlos. (2012). Mathematical Models in Population Biology and Epidemiology (2nd ed.).
	
%	\bibitem{Brauer 2017} Brauer, F. (2017). Mathematical epidemiology: Past, present, and future. Infectious Disease Modelling, 2(2), 113–127. https://doi.org/10.1016/j.idm.2017.02.001.
	
%	\bibitem{Cartwright 1972} Cartwright, F. F., \& Biddiss, Michael D. (1972). Disease and history. London: Hart-Davis.
	
	\bibitem{Cheng 2017} Cheng A.C., Holmes M., Dwyer D.E., et al. (2017). Influenza epidemiology in patients admitted to sentinel Australian hospitals in 2016: the Influenza Complications Alert Network (FluCAN). Commun Dis Intell Q Rep;41(4):E337‐E347. Published 2017 Dec 1.
	
	\bibitem{Clothier 2006} Clothier, H. J., Atkin, L., Turner, J., Sundararajan, V., \& Kelly, H. A. (2006). A comparison of data sources for the surveillance of seasonal and pandemic influenza in Victoria. Communicable Diseases Intelligence Quarterly Report, 30(3), 345–349.
		
	\bibitem{Daley Gani 2001} Daley D.J., Gani J. (2001). Epidemic Modelling: An Introduction (Cambridge Studies in Mathematical Biology) Cambridge: Cambridge University Press.
	
	\bibitem{Dep of Health 2020} Australian National Notifiable Diseases and Case Definitions, Department of Health, Australia. National Surveillance Case Definitions for the Australian National Notifiable Diseases Surveillance System. Available from: https://www1.health.gov.au/internet/main/publishing.nsf/Content/cdna-casedefinitions.htm Accessed on 7 October 2020.
	
%	\bibitem{Diekmann et al 1990} Diekmann, O., Heesterbeek, J., \& Metz, A. (1990). On the definition and the computation of the basic reproduction ratio R 0 in models for infectious diseases in heterogeneous populations. Journal of Mathematical Biology, 28(4), 365–382. https://doi.org/10.1007/BF00178324
	
	\bibitem{Dietz Heesterbeek 2002} Dietz, K., \& Heesterbeek, J. A. (2002). Daniel Bernoulli’s epidemiological model revisited. Mathematical Biosciences, 180(1), 1–21. https://doi.org/10.1016/S0025-5564(02)00122-0.
	
	\bibitem{Geard et al 2013} Geard, N., McCaw, J., Dorin, A., Korb, K., \& McVernon, J. (2013). Synthetic population dynamics: a model of household demography. Journal of Artificial Societies and Social Simulation, 16(1). https://doi.org/10.18564/jasss.2098.
	
	\bibitem{Gelman et al 2014} Gelman, A., Carlin, John B., Stern, Hal S., Dunson, David B., Vehtari, Aki, \& Rubin, Donald B. (2014). Bayesian data analysis (Third edition.).
	
%	\bibitem{Gibson Bruck 2000} Gibson, M. A., \& Bruck, J. (2000). Efficient Exact Stochastic Simulation of Chemical Systems with Many Species and Many Channels. The Journal of Physical Chemistry A, 104(9), 1876–1889. https://doi.org/10.1021/jp993732q.
	
	\bibitem{Gillespie 1976} Gillespie, D. T. (1976). A general method for numerically simulating the stochastic time evolution of coupled chemical reactions. Journal of Computational Physics, 22(4), 403–434. https://doi.org/10.1016/0021-9991(76)90041-3.
	
	\bibitem{Gillespie 2001} Gillespie, D. (2001). Approximate accelerated stochastic simulation of chemically reacting systems. Journal of Chemical Physics, 115(4), 1716–1733. https://doi.org/10.1063/1.1378322.
	
	\bibitem{Hamer 1929} Hamer, S. W. (1929). Epidemiology, Old and New. Southern Medical Journal, 22(5). https://doi.org/10.1097/00007611-192905000-00036.
	
	\bibitem{Held 2020} Held, L., Hens, Niel, O'Neill, Philip D, Wallinga, Jacco, \& ProQuest. (2020). Handbook of infectious disease data analysis.
	
%	\bibitem{Hethcote 2000} Hethcote, H. W. (2000). The Mathematics of Infectious Diseases. SIAM Review, 42(4), 599–653. https://doi.org/10.1137/S0036144500371907.
	
%	\bibitem{House et al 2011} House, T., Baguelin, M., Van Hoek, A. J., White, P. J., Sadique, Z., Eames, K., Keeling, M. J. (2011). Modelling the impact of local reactive school closures on critical care provision during an influenza pandemic. Proceedings of the Royal Society B, 278(1719), 2753–2760. https://doi.org/10.1098/rspb.2010.2688.
	
%	\bibitem{Jenkinson Goutsias 2012} Jenkinson, G., Goutsias, J. (2012). Numerical integration of the master equation in some models of stochastic epidemiology. PLoS ONE 7, e36160.
	
	\bibitem{Keeling Rohani 2007} Keeling, M.J., Rohani, P. (2007) Modeling Infectious Diseases in Humans and Animals. Princeton University Press.
	
	\bibitem{Kermack McKendrick 1927} Kermack W.O., McKendrick A.G. (1927) Contribution to the mathematical theory of epidemics. Proc R Soc Lond A, 115, 700–721.
	
%	\bibitem{Kiple 1993} Kiple, K. F. (1993). The Cambridge world history of human disease.
	
	\bibitem{Kroese et al 2011} Kroese, D. P. P., Taimre, T. I., \& Botev, Z. I. (2011). Handbook of Monte Carlo Methods (pp. 1–752). Wiley Blackwell. https://doi.org/10.1002/9781118014967.
	
	\bibitem{Leung 2015} Leung, N. H. L., Xu, C. M., Ip, D. K. J., \& Cowling, B. (2015). Review Article: The Fraction of Influenza Virus Infections That Are Asymptomatic: A Systematic Review and Meta-analysis. Epidemiology, 26(6), 862–872. https://doi.org/10.1097/EDE.0000000000000340.
	
%	\bibitem{MacDonald 1952} Macdonald, G. (1952). The analysis of equilibrium in malaria. Tropical Diseases Bulletin. 49 (9): 813–829. ISSN 0041-3240. PMID 12995455.
	
	\bibitem{Mclean et al 2010} McLean, Pebody, Campbell, Chamberland, Hawkins, Nguyen-Van-Tam, Watson. (2010). Pandemic (H1N1) 2009 influenza in the UK: clinical and epidemiological findings from the first few hundred (FF100) cases. Epidemiology and Infection, 138(11), 1531–1541. https://doi.org/10.1017/S0950268810001366.
	
%	\bibitem{Freund 2014} Miller, I., Miller, Marylees, \& Freund, John E. (2014). John E. Freund's Mathematical Statistics with Applications. (8th ed; international ed.). Pearson.
	
	\bibitem{Moss et al 2016} Moss, R., Zarebski, A., Dawson, P., \& McCAW, J. M. (2017). Retrospective forecasting of the 2010–2014 Melbourne influenza seasons using multiple surveillance systems. 145(1), 156–169. https://doi.org/10.1017/S0950268816002053.
	
	\bibitem{McCaw et al 2018} McCaw, J. M., Moss, R., Shearer, F., Lau, T., Dawson, P. EpiDefence and EpiFX. Websites https://eng.unimelb.edu.au/industry/defence/capabilities/medical-\\
	countermeasures/case-studies/epifx-and-epidefend and 
	https://www.dst.defence. gov.au/news/2016/07/12/bio-terrorism-algorithm-forecasting-flu-outbreaks.
	
	\bibitem{Moss et al 2019} Moss, R., Zarebski, A. E., Carlson, S. J., \& Mccaw, J. M. (2019). Accounting for Healthcare-Seeking Behaviours and Testing Practices in Real-Time Influenza Forecasts. Tropical Medicine and Infectious Disease, 4(1). https://doi.org/10.3390/tropicalmed4010012.
	
	\bibitem{Murray 1989} Murray, J. D., Murray, James D, \& SpringerLink. (1993). Mathematical biology (Second, corrected edition.). Springer-Verlag.
	
	\bibitem{Patrozou Mermel 2009} Patrozou, E., \& Mermel, L. A. (2009). Does influenza transmission occur from asymptomatic infection or prior to symptom onset? Public health reports (Washington, D.C. : 1974), 124(2), 193–196. https://doi.org/10.1177/003335490912400205.
	
	\bibitem{Ristic Dawson 2013} Ristic, B., \& Dawson, P. (2016). Real-time forecasting of an epidemic outbreak: Ebola 2014/2015 case study. 2016 19th International Conference on Information Fusion (FUSION), 1983–1990. ISIF.
	
	\bibitem{Ross 1911} Ross, R. (1911) The Prevention of Malaria. John Murray, London.
	
	\bibitem{Ross et al 2010} Ross, J.V., House, T., Keeling, M.J. (2010) Calculation of Disease Dynamics in a Population of Households. PLoS ONE 5(3): e9666. doi:10.1371/journal.pone.0009666.
	
	\bibitem{Scragg 1985} Scragg, R. (1985). Effect of influenza epidemics on Australian mortality. The Medical Journal of Australia, 142(2), 98–102. https://doi.org/10.5694/j.1326-5377.1985.tb133043.
	
	\bibitem{Spiegelhalter 2019} Spiegelhalter, D. J. (2019). The art of statistics: learning from data.
	
%	\bibitem{Tadros Hewitt 2020} Tadros, E. \& Hewitt, L. (2020). COVID-19: Tracking the economic recovery. Retrieved from https://www.afr.com/policy/economy/covid-19-the-road-to-recovery-20200527-p54wz6
	
	\bibitem{Thomas et al 2015} Thomas, E. G., McCaw, J. M., Kelly, H. A., Grant, K. A., \& McVernon, J. (2015). Quantifying differences in the epidemic curves from three influenza surveillance systems: a nonlinear regression analysis. 143(2), 427–439. https://doi.org/10.1017/S0950268814000764.
	
	\bibitem{Walker et al 2017} Walker, J.N., Ross J.V., Black A.J. (2017) Inference of epidemiological parameters from household stratified data. PLoS ONE 12(10): e0185910. https://doi.org/10.1371/journal.pone.0185910.
	
	\bibitem{Zarebski et al 2017} Zarebski, A. E., Dawson, P., McCaw, J. M., \& Moss, R. (2017). Model selection for seasonal influenza forecasting. Infectious Disease Modelling, 2(1), 56–70. https://doi.org/10.1016/j.idm.2016.12.004.
	
	
	
\end{thebibliography}