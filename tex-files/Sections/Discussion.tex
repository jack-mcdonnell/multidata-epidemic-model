
% Discussion
An initial attempt to estimate the parameters included data from all surveillance systems but failed to produce a result. Further analysis revealed that by omitting the case notification data this method was able to estimate the parameters with fair accuracy and efficiency. This outcome was unexpected since the data for case notifications were more abundant than both either FluTracking or FluCAN and should have had more explanatory power. This suggests that the issue was due to an error either in the formulation of the likelihood, or somewhere in the implementation, from the generation of the data using the household SEEIIHR epidemic model to the code that computed the log-likelihood. An exploratory analysis failed to identify the cause of this issue. More work is needed to resolve this issue. The reason for using a Bayesian framework is to be able to update the parameter estimates as more data become available. The results presented here used all of the data that had been collected by the end of the 98 day period. Future work should examine the ability of this method to update the parameter estimates each week.

\subsection{Implications}
The results presented in the previous chapter show that it is possible to formulate an effective and efficient Bayesian inference framework for a household epidemic model that integrates multiple sources of data. Notwithstanding the issues with the case notification data, this represents an important first step towards a general method for integrating multiple sources of surveillance data into a single parameter estimation framework. There was not enough time during this project to compare the results of the multidata approach with a more traditional single parameter approach. If the issues outlined above can be resolved, the next step would be to undertake a quantitative comparison of the performance of this approach with a single-parameter approach to assess the benefits from integrating multiple data sources.

\subsection{Limitations}
There were many limitations of the methods described in Chapters \ref{chp: A Multi-Data Household Epidemic Model} and \ref{chp: Parameter Estimation}, and the results presented in Chapter \ref{chp: Results}. Some of these are discussed here.\\
The household structure used to generate the results in the previous chapter consisted of 100 households each with 1,000 members. Although this is not a realistic household structure, it offered a substantial benefit in run time while allowing disease transmission within and between households. A more realistic structure might be 100,000 households each with say four members. Given more time and a faster implementation, a more realistic household structure may be simulated using the methods described here.\\
The observation model accounted for the sensitivity of the laboratory tests, i.e. the true positive rate $p_+$. This meant that the number of positive non-hospital tests ($R_+^G$) depended only on the number of cases reporting being tested ($R_c^G$) and the sensitivity ($p_+$). However these tests may also have a specificity $<1$, i.e. a non-zero false positive rate. It would be possible to account for false positives by assuming some specificity $p_{spec} > 0$ and letting $R_+^G$ also depend also on the number of non-influenza ILI cases that return positive tests, i.e. $R_+^G | R_c^{Gf}, R_c^{Gnf} \sim binom(R_c^{Gf}, p_+) + binom(R_c^{Gnf}, 1-p_{spec})$. \\
The results described in Chapter \ref{chp: Results} were generated using only two of the four surveillance systems originally included in the formulation of this problem. The First Few Hundred data were purposely left out of the formulation in Chapter \ref{chp: A Multi-Data Household Epidemic Model}, and future work should attempt to resolve the issues that prevented the First Few Hundred and case notification data from being included in this implementation. \\
Despite the substantial increase in speed granted by the $\tau$-leap stochastic simulation algorithm, there is no doubt potential to increase the efficiency of the epidemic simulation. Possible improvements include using a faster exact method, or an algorithm that can switch dynamically between the exact and approximate methods when the number of infected individuals crosses a critical threshold $I_{crit}$, i.e. when $I < I_{crit}$ use the direct method, when $I>I_{crit}$ switch to $\tau$-leaping, and \textit{vice versa}.\\
Although much thought and effort went into an efficient implementation of the household epidemic model, there are still opportunities to improve the efficiency further. These include improving existing code with faster algorithms, and using a faster language such as C++.

\subsection{Future Work}
The research presented here represents a first step towards developing Bayesian methods for estimating the parameters of an epidemic model using data from multiple surveillance systems simultaneously. Consequently, much more work is required in order to develop these methods into a robust framework that can be adapted to different diseases and surveillance systems. Furthermore, it must also offer an advantage over traditional methods in terms of forecasting timeliness and accuracy. In addition to recommendations for future work already noted in this section, some potential topics of future work include the following.
The observation model and likelihood could be extended to other sources of data such as sewage samples, Google Flu Trends, smart-phone contact tracing applications, and social media posts. Selected data sources should complement each other in terms of time (for early detection) or other aspects. 
The population model used in the epidemic simulation was quite simple. More complexity, and hopefully realism, could be attained by using a more complex population model, such as an agent-based approach with agents that have demographic information, e.g. name, age, population changes over time (births, deaths, moving household etc.). 
The recent (and continuing) pandemic of the novel coronavirus SARS-COV-2 has lead to substantial changes in the public health response, such as data collection and contact tracing. For example, new case notifications are reported daily instead of weekly. This wealth of data could provide a valuable opportunity to assess these new multi-data methods.
\\ \\
This section provided a short discussion of the results that were presented in the previous chapter, some implications and limitations of the approach described in this thesis, and potential future work to build on the research undertaken during this research.
