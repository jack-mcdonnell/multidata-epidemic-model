
% Conclusion
The goal of this study was to develop a proof of concept for a Bayesian parameter estimation framework for an epidemic model that integrates multiple sources of data. This thesis documented the motivation, construction, and implementation of such a framework, and concluded with a discussion on the work presented here and direction for potential future research. \\
A literature review provided the background and requisite knowledge, including some modern epidemic modelling techniques, parameter estimation using the Metropolis-Hastings algorithm, and the recent research that generated the motivation for this project. A household epidemic model that integrates multiple surveillance systems originally proposed during a MATRIX workshop in 2019 was constructed and analysed. This included the epidemic transmission model and observation model for surveillance data, as well as a computationally efficient method for simulating the epidemic. Bayesian inference methods included constructing the likelihood, computing the logarithm of the likelihood, and efficiently implementing the Metropolis-Hastings algorithm. The results of this implementation demonstrated the accuracy and efficiency of the Bayesian framework to estimate six epidemic transmission parameters, from which it is possible to estimate other key epidemiological quantities that inform public health decision-making, such as effective reproduction number.\\
A detailed analysis of results could not be performed due to time constraints, and there is still much to investigate regarding the performance of this multi-data Bayesian framework. Potential future topics of research include resolving the issues encountered in this project, improving the efficiency of both the epidemic model and the parameter estimation, and quantitatively assessing the advantages of this approach over more traditional approaches.\\
Despite public health intervention measures such as vaccination programmes, seasonal influenza continues to be an annual cause of morbidity and mortality worldwide. Notwithstanding the abundance of surveillance data, forecasting the severity and duration of seasonal influenza continues to challenge epidemiologists each year.\\
As technology continues to evolve, the variety and volume of epidemic surveillance data continues to increase, from internet search queries to social media posts and smart-phone contact-tracing applications. Effective inference methods that can integrate these sources of data promise to provide a means of improving the timeliness and accuracy of epidemic forecasts and the evaluation of intervention measures. The work presented in this thesis represents an important first step towards developing methods that can integrate complex epidemic surveillance data.
\\ Although the context of this research was seasonal influenza, these methods are adaptable to other infectious diseases and surveillance systems. At the time of writing, the pandemic outbreak of a novel coronavirus SARS-COV-2 has infected over forty-five million people, resulting in over a million deaths, and unprecedented social and economic disruption. Statistical methods that provide early insight into the spread of epidemics will only become more important for managing the public health response, now and in the future.