
\noindent Mathematical modelling can inform public health decision making by forecasting the effects of an epidemic and evaluating intervention measures, such as vaccination programmes.
Numerous surveillance systems collect data that can be used in conjunction with epidemic models to provide estimates of model parameters. However, traditional approaches tend to use models that consider single sources of data. Previous attempts to combine multiple sources of data simultaneously have either resulted in competing parameter estimates or ignored some of the available information.
In this paper, we present a household epidemic model that integrates data from multiple surveillance systems, and an approach to Bayesian inference that includes explicit computation of the likelihood and is implemented using the Metropolis-Hastings algorithm. The code for the results presented here can be found at 
https://github.com/jack-mcdonnell/multidata-epidemic-model